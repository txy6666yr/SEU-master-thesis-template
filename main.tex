% \documentclass[algorithmlist,figurelist,tablelist,nomlist]{template/seumasterthesis} %学硕使用这一行
\documentclass[algorithmlist,figurelist,tablelist,nomlist,engineer]{template/seumasterthesis}
\usepackage{multirow} % 处理跨行表格数据
\usepackage{float}
\usepackage{lipsum}
\usepackage{pifont}  % 引入 pifont 宏包
\usepackage{enumitem}
% Mac系统取消以下两行代码的注释,
% \usepackage{fontspec} % 加载 fontspec 宏包
% \usepackage{xeCJK}    % 加载 xeCJK 宏包以支持中文

% Mac系统如需更改宋体与Windows保持一直,前往https://github.com/Reanon/SEUThesisLatexTemplate/tree/master/font下载Simsun.ttf后打开安装到系统中即可

\renewcommand{\thefootnote}{\ding{\numexpr171+\value{footnote}}}
\setCJKmainfont{simsun.ttc}[AutoFakeBold]  %使用Overleaf需要注释掉这一行
\begin{document}

%% ----------------------------------------------------------------------------
%%                                 Meta Data
%% ----------------------------------------------------------------------------
\categorynumber{TN92} %《中国图书资料分类法》分类法
\UDC{621.3}              %《国际十进分类法UDC》的类号
\secretlevel{公开}        % 学位论文密级分为"公开"、"内部"、"秘密"和"机密"四种
\studentid{220000}      % 学号要完整,前面的零不能省略

%% ----------------------------------------------------------------------------
%%                           Thesis Title and Spine
%% ----------------------------------------------------------------------------
\title
    {东南大学 \LaTeX 论文模板使用手册}        % 论文中文标题
    {如何优雅地撰写硕士研究生毕业论文}         % 论文中文副标题,没有可以空着
    {Southeast University \LaTeX ~Thesis Template User Manual}  % 论文英文标题
    {How to Write a Master Thesis in an Elegant Way}            % 论文英文副标题,没有可以空着

\spine
	% 书脊标题与副标题
    {东南大学 \rotatebox{270}{\raisebox{2.5pt}{LaTeX}} 论文模板使用手册} 
    {}                                                               

%% ----------------------------------------------------------------------------
%%                             Author and Advidor
%% ----------------------------------------------------------------------------
\author
    {王东南}                        % 作者中文姓名
    {XX Xxxx}                  % 作者英文姓名,首字母大写,姓名分开,双字用「-」连接

\advisor
    {王东南 教授}                % 导师中文姓名
    {YY Yyyy}        % 导师英文姓名
    {Prof.}                     % 导师职称
    
\coadvisor                 % 联合培养导师姓名,没有可以不写
    {王东南 高工}                  % 导师中文姓名
    {ZZ Zzzz}             % 导师英文姓名
    {Eng.}                 % 导师职称 (English), 如教授(Prof.)、副教授(A.P.)
%% ----------------------------------------------------------------------------
%%                              Thesis Defence
%% ----------------------------------------------------------------------------
\engthesistype{应用研究}            % 工程硕士论文类型
\degreetype                        % 学位类型
    {电子信息硕士}
    {Master of Electronic Information}
\major{ \begin{minipage}{\linewidth}\centering 通信工程(含宽带\\网络、移动通信等)\end{minipage}}                 % 一级学科名
\submajor{通信与信息系统}             % 二级学科名
\defenddate{2025年5月18日}          % 答辩日期 \today
\authorizedate{}                  % 授予学位日期,这个档案袋不需要填
\committeechair{}               % 答辩委员会主席姓名
\reviewer{}{}                % 两位论文评阅人姓名
\department                        % 学院名称
    {信息科学与工程学院}
    {School of Information Science and Engineering}
% \seuthesisthanks                % 资助信息,没有可以不写
%     {本文的部分工作受国家自然基金 No. zxgg666 的支持与帮助,在此表示感谢。}

%garbagecoder: \let 是 TeX 的赋值命令,它会让 \cleardoublepage 变成 \clearpage,原模板限制奇偶页适用于双面打印,此命令用于去掉多余的空白页
% \let\cleardoublepage\clearpage 

%% ----------------------------------------------------------------------------
%%                                  Cover
%% ----------------------------------------------------------------------------
% ⚠️ 可以在编写论文的时候注释掉封面,加快编译速度
\makebigcover  % 生成A3大封面
\makecover     % 生成小封面
	 
%% ----------------------------------------------------------------------------
%%                          Abstract and Contents
%% ----------------------------------------------------------------------------
\input{chapters/abstract}
\cleardoublepage
\addcontentsline{toc}{chapter}{目  录}
\tableofcontents          % 生成目录
\listofothers             % 生成图、表等目录,没有可以不写
 
%% ----------------------------------------------------------------------------
%%                                Main Body
%% ----------------------------------------------------------------------------
\mainmatter                    % 开始正文
\input{chapters/chapter1}      % 第一章:
\input{chapters/chapter2}      % 第二章:
\input{chapters/chapter3}      % 第三章:
\input{chapters/chapter4}      % 第四章:
\input{chapters/chapter5}      % 第五章:
\input{chapters/chapter6}      % 第六章:

%% ----------------------------------------------------------------------------
%%            Acknowledgement, Appendix, Bibliography and Resume
%% ----------------------------------------------------------------------------
\input{chapters/acknowledgement}    % 致谢
\thesisbib{IEEEfull,reference}               % 生成参考文献

%% 下面一句只是用于提示 TexPad 参考文献位置,正式生成时一定要删除
% \bibliography{reference.bib} % 告诉编译器参考文献所在文件

\input{chapters/appendix}           % 附录
\input{chapters/resume}             % 作者简介
\clearpage
\thispagestyle{empty}
\begin{center}
    \xiaoerhao\songti\bfseries 毕业/学位论文答辩委员会名单
\end{center}
% Please add the following required packages to your document preamble:
% \usepackage{multirow}
\begin{table}[h]
\renewcommand\arraystretch{1.3}
\centering
\begin{tabular}{|cc|ccc|}
\hline
\multicolumn{2}{|c|}{\textbf{毕业/学位论文题目}}    & \multicolumn{3}{l|}{东南大学 LATEX 论文模板使用手册}  \\ \hline
\multicolumn{2}{|c|}{\textbf{作  者}} & \multicolumn{3}{l|}{XXX} \\ \hline
\multicolumn{2}{|c|}{\textbf{专  业}} & \multicolumn{3}{l|}{通信工程(含宽带网络、移动通信)等}        \\ \hline
\multicolumn{2}{|c|}{\textbf{研究方向}} & \multicolumn{3}{l|}{通信与信息系统}\\ \hline
\multicolumn{2}{|c|}{\textbf{导  师}} & \multicolumn{3}{l|}{XXX}  \\ \hline
\multicolumn{1}{|c|}{\multirow{7}{*}{\bfseries \parbox[t]{1em}{答辩委员会组成}}} & \textbf{姓  名}   & \multicolumn{1}{c|}{\textbf{职  称}}        & \multicolumn{1}{c|}{\textbf{学科专业}} & \textbf{工作单位} \\ \cline{2-5} 
\multicolumn{1}{|c|}{}    & \multicolumn{1}{c|}{\renewcommand\arraystretch{1}\begin{tabular}[c]{@{}c@{}}XXX\\ (主席)\end{tabular}} & \multicolumn{1}{c|}{研究员}  & \multicolumn{1}{c|}{ 信息与通信工程 }       & XX大学\\ \cline{2-5} 
\multicolumn{1}{|c|}{}    & XXX     & \multicolumn{1}{c|}{\renewcommand\arraystretch{1}\begin{tabular}[c]{@{}c@{}}研究员级\\ 高级工程师\end{tabular}}   & \multicolumn{1}{c|}{信息与通信工程}       &   XXX实验室   \\ \cline{2-5} 
\multicolumn{1}{|c|}{}    & \multicolumn{1}{c|}{\renewcommand\arraystretch{1}\begin{tabular}[c]{@{}c@{}} XXX\\ XXX\end{tabular}}      & \multicolumn{1}{c|}{副教授}  & \multicolumn{1}{c|}{信息与通信工程}       &    \multicolumn{1}{c|}{\renewcommand\arraystretch{1}\begin{tabular}[c]{@{}c@{}}University of XXX\\ XXX\end{tabular}}     \\ \cline{2-5} 
\multicolumn{1}{|c|}{}    & XXX     & \multicolumn{1}{c|}{副研究员}  & \multicolumn{1}{c|}{信息与通信工程}       & XX大学\\ \cline{2-5} 
% \multicolumn{1}{|c|}{}    &         & \multicolumn{1}{c|}{}    & \multicolumn{1}{c|}{}   &    \\ \cline{2-5} 
\multicolumn{1}{|c|}{}    & \multicolumn{1}{c|}{\renewcommand\arraystretch{1}\begin{tabular}[c]{@{}c@{}}XXX\\ (秘书)\end{tabular}}   & \multicolumn{1}{c|}{副教授} & \multicolumn{1}{c|}{信息与通信工程}       & XX大学\\ \hline
\end{tabular}
\end{table}


备注:

1、本表格适用于所有研究生。

2、本表格排版在终版毕业/学位论文中,附在毕业/学位论文的最后。
    % 毕业/学位论文答辩委员会名单
\end{document}
